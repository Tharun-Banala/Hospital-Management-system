\documentclass{scrreprt}
\usepackage{listings}
\usepackage{underscore}
\usepackage{graphicx}

\usepackage[bookmarks=true]{hyperref}
\usepackage[utf8]{inputenc}
\usepackage[english]{babel}
\hypersetup{
    bookmarks=false,    % show bookmarks bar?
    pdftitle={Software Requirement Specification},    % title
    pdfauthor={Jean-Philippe Eisenbarth},                     % author
    pdfsubject={TeX and LaTeX},                        % subject of the document
    pdfkeywords={TeX, LaTeX, graphics, images}, % list of keywords
    colorlinks=true,       % false: boxed links; true: colored links
    linkcolor=blue,       % color of internal links
    citecolor=black,       % color of links to bibliography
    filecolor=black,        % color of file links
    urlcolor=purple,        % color of external links
    linktoc=page            % only page is linked
}%
\def\myversion{1.0 }
\date{}
%\title
\usepackage{hyperref}
\begin{document}

\begin{flushright}
    \rule{16cm}{5pt}\vskip1cm
    \begin{bfseries}
        \Huge{SOFTWARE REQUIREMENTS\\ SPECIFICATION}\\
        \vspace{1cm}
        for\\
        \vspace{1cm}
        HMS WEBSITE\\
        \vspace{1cm}
        \LARGE{Version \myversion}\\
        \vspace{1cm}
        Prepared by : \\
        Boddupalli Karthik (200001016)\\
        K Sreekara Madyastha (200001031)\\
        Byri Sahas Reddy (200001017) \\
        Banala Tharun (200001013) \\
        Shaik Wanhar Aziz (200001072) \\
        \vspace{1.5cm}
        Submitted to : Dr. Puneet Gupta \\Professor\\
        \vspace{1.5cm}
        \today\\
    \end{bfseries}
\end{flushright}

\tableofcontents

\chapter{Introduction}

\section{Purpose}
In a hospital, documents like information of patients and their appointments with doctors, medical reports, lab reports, prescriptions, billings, emergencies, staff details, ..etc are very important and must be stored and maintained by both the hospital and by the patients. In case of physical documents/reports, it's not uncommon that they might be lost or even worse altered and there's nothing one can do about it and the most difficult thing to do is to maintain all those records in the form of papers. The solution to all those problems is ``HMS Website''. The purpose of this website is to solve the complications coming from managing all the paper works of every patient associated with the various departments of hospitalization with confidentiality. HMS provides the ability to manage all the paperwork in one place, reducing the work of staff in arranging and analyzing the paperwork of the patients. With that, now patients, hospital staff and administration can view, write/upload, update and print those documents/reports anytime-anywhere. 

\section{Intended Audience and Reading Suggestions}
This SRS is for developers, project managers, users and testers. Further the discussion will provide all the internal, external, functional and also non-functional informations about ``HMS WEBSITE''.

\section{Project Scope}
``HMS Website'' can be used to access all the necessary information that a patient, doctor or hospital staff go through in their time within the hospital. \\
\\
Let's say a patient comes to the hospital with a health problem. First, the receptionist receives them by creating a personal account for the patient (if this is their first visit) and makes an appointment with a doctor of concerned field and store it in the website. Now, that doctor and his/her medical assistant will receive a notification about this patient's appointment. When this patient goes to the doctor, the medical assistant creates a report for this patient by writing the patient's problem, his/her observations, doctor's observations and suggestions, prescriptions, any tests recommended by doctor and other relevant information using this website. The patient can view all these reports anytime by logging in with their credentials. If any tests were recommended by the doctor, the medical assistant assigns this patient to a diagnostician and they will also be able to view the required information about the patient. Once the patient reaches the lab and the diagnostician performs the tests, those test/lab reports will be uploaded in the website and are made available for the doctor. medical assistant and the patient to see. If the patient is willing to stay in the ward, the medical assistant assigns this patient to a ward clerk and they will also be able to view the required information about the patient. Throughout their stay in the ward, the ward clerk maintains the record of the patient's condition, visits and any other necessary information. When a patient/ward clerk goes to the pharmacy to purchase medicines, the pharmacy technician can view the patient's prescriptions using the unique PatientID and generate a bill for the medicines purchased that can be added to the patient's total hospital bill or can be paid seperately. And at the billing desk, the biller maintains all the payments made in the hospital and can also see the patient's health schemes given their PatientID. Finally, the administration can view all this information which helps monitoring the hospital and making it easier for everyone.
\chapter{Overall Description}

\section{Product Perspective}
The Hospital Management system is a self-contained digital system that manages various activities like maintaining details of patients and their treatment, ward assignments, lab and medicine records, billing etc. to ensure smooth working of a hospital. It provides much better reliability, efficiency, security and performance than the manual file system.
It keeps track of all data that was or is in use, and displays them to the concerned personnel, with access privilege, without any requirements of external intervention.

It provides a clear and clean interface that minimizes the work by facilitating convenient management of activities of the medical centre.

\section{User Classes and Characteristics}
The hospital management system consists of four user classes:

\begin{itemize}
  \item Patients: Add their details, view their appointments, reports, bills.
  \item Admin: Admin is the highest privileged user who will be able to manage any account or activity of the system.
  \item Medical Staff
    \begin{itemize}
        \item Doctors: Keep track of patients' diagnoses and treatment. 
        \item Medical assistants: Generate and keep track of patients' medical reports.
        \item Diagnosticians: Generate lab reports and keep track of patients' test details.   
    \end{itemize}
    
    \item Non-Medical Staff
    \begin{itemize}
        \item Receptionists: Schedule appointment details and view general hospital information.
        \item Ward assistants: Manage ward information.
        \item Billing staff: Generate and keep track of patients' bills. 
    \end{itemize}

\end{itemize}
The admin provides the login id and password for all the staff. The software is developed to make all the processes user friendly. The users are expected to have the basic computer knowledge to use the system. Staff may need to be trained to use the system effectively.
\section{Product Functions}
The Hospital management system will store required information of patients,doctors or any staff like medical and diagnostic reports,bills,patient and ward data.\\
\newline
It facilitates privacy for every user by making unique identification of them using their username,password, role.\\
\newline
It also stores personal information like first name,last name, blood group,age,email and permanent address,salary if staff etc.any 
required or relevant information can be added.\\
\newline
The system supports doctor by updating all the information about their current patient and shows all their past patient and their reports.that is by displaying patients name, patient ID, Issue, diagnostics report and prescription.\\
\newline
It allows ward clerk to keep track of all patients by name and patient ID in his/her ward and their medication if needed. It also keeps record of patient bills for billing desk.\\
\newline
On above all administration desk can see all the patient info that is patient name,ID,number of visits etc and all the staff information namely ID,name,shift,salary etc.\\
\newline 
The diagnosis data will include patient ID, reports, prescribed doctor ID and diagnostician ID. Billing data will have patient id, and his/her bills, their status,date paid if paid, and Appointment ID along with visit number for that appointment.\\
\newline
all these while making sure that any information does not end up to anyone that is not supposed to reach.\\
\newline
System allows a pharmacy by storing all the data medicines and their bills along with patient ID\\
\newline
E.R Diagram of the system is shown in figure \ref{fig:Entity Relation Diagram} below\\
\newline
*The shown ER diagram may change in future due to change in requirements or in Development Phase. This is just a prototype of original ER\\ diagram
\newline

\section{Operating Environment}
The website will be operate in any Operating Environment - Mac, Windows, Linux etc with a requirement of a web browser. 

\section{Design and Implementation Constraints}

\begin{itemize}
    \item All the information about users and reports must be kept in a database that the website can access.
    \item To access their online accounts and perform operations, users must provide the correct usernames and passwords.
    \item Be able to manage a large number of transactions (including concurrent) at any time.
    \item Make information available at all times, especially during concurrent requests.
    \item Even during parallel requests, the integrity of the information should be ensured.
    \item This project will include sensitive medical and personal information. As a result, when designing this project, access privileges, security features, and login fail-safes will be of the highest concern.
    \item The system should be user-friendly.
\newpage
\end{itemize}
\begin{figure}
    \centering
    \includegraphics[width=14cm]{er.jpeg}
    \caption{Entity Relation Diagram.}
    \label{fig:Entity Relation Diagram}
\end{figure}
\chapter{System Features}

Hospital Management System (HMS) is powerful, flexible, and easy to use system designed for a hospital environment. Hospital Management System
eases the wide range of administration and management process of a hospital( mainly multi-speciality hospital). It is an
integrated end-to-end Hospital Management System (HMS) that provides relevant
information across the hospital to support effective decision making for patient
care, hospital administration and critical financial accounting, in a seamless flow.

\section{Description and Priority}
Hospital Management System has wide range of features. Few of the main features are prioritized up to down.
 
\begin{enumerate}
    \item Appointment Schedule :It is the main feature of HMS. It is been operated by receptionists to schedule doctor's appointments with new and follow-up patients.
     \item Information Access :  Staff can access their related information based on their job to smoothen the management process. Such as  Patient's personal information and medical information are accessed by authorised doctors and respected staffs. Doctor's information are accessed by receptionists and medical assistants. etc..
    \item Bill Generation : This is done by billing desk department, pharmacy technician and ward clerks to generate and update the hospital bills, pharmacy bills and ward bills respectively.
    \item Report generation : This feature is used to generate/upload medical and lab reports which are later accessed by patients. It is been operated by diagnosticians,doctors and medical-assistants.
    \item Administration : It is done by administrator to manage/monitor the overall information.
    \item Information portal : All the patients who have an account in the hospital can view their personal and medical information.
    \item Download Access : Patients have the access to download/view their reports.
\end{enumerate}

\section{Functional Requirements}
The "HMS" website is being build on Django,HTML,CSS,Javascript,React,Bootsrap,Tailwind and MySQL.\\
\newline
Back-End - Django (python framework).\\
\newline
Font-End - React,JavaScript,HTML,CSS,Bootsrap, Tailwind CSS .\\
\newline
Database -  MySQL.

\chapter{Other Nonfunctional Requirements}

\section{Performance Requirements}
"Hospital Management system portal" will be used for maintaining the patient records and providing other facilities to accommodate patient's stay at hospital. 
\newline
At the time of getting vital information especially during the times of emergency, the system should retrieve information quickly.
So for more interaction React and MySQL is used. 

\section{Security Requirements}
The system needs the users to recognize themselves using their login credentials.
One particular user of a section only can perform his/her particular actions.
\newline
Users cannot view or edit the data if they don’t have the specific authorizations to ensure Doctor-Patient confidentiality (View level data abstraction).
\newline
Changing of profile information will require One time password confirmation,

\section{Software Quality Attributes}
\textbf{Availability}- The system should be available all the time.
\newline
\textbf{Correctness}- System should not allow any duplicates, which may lead to wrong treatments or any other mishap.
\newline
\textbf{Reliability}- The system should not hang under any circumstance.
\newline
In the development phase also testing and conferences of users is been continued. So that the quality of the software is been maintained and all the requirements are been fulfilled.
\newline
\textbf{Maintainability}- Also the system is able to accommodate changes or incorporate new features according to the hospital's needs.
\newline
The system will track every mistake as well as keep a log of it.
\newline
Database, logical and also UI test is required. 

\section{Business Rules}
Business Rules are designed based on the regulation by the governments and information storage and access are done based on the regulatory guidelines.


\chapter{Other Requirements}
"Hospital Management System" needs maintenance as it is a long process software. It will need re-factoring and further the requirements can be changed as the field is changing frequently.


\end{document}